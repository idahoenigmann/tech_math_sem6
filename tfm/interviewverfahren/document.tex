\documentclass[]{article}
\usepackage[a4paper, margin=3cm]{geometry}
\usepackage{enumitem}
\usepackage{amssymb}
\usepackage{pdflscape}
\usepackage{tabularx}


\setlength\extrarowheight{2pt}

%opening
\title{Interviewverfahren\\ Selbstbedienungs-Ticketautomaten ÖBB}
\author{Ida Hönigmann}

\begin{document}

\maketitle

\begin{abstract}
	TODO Abstract
\end{abstract}

\section{Fragestellungen}
\label{sec:fragestellungen}

Ziel der Befragung ist es, die Beweggründe zur Wahl unterschiedlicher Kaufoptionen von ÖBB Zugtickets zu erforschen. Besonderer Fokus soll auf die Unterschiede zwischen den Selbstbedienungs-Automaten und dem bedienten Ticketschaltern liegen.

Als mögliche Beweggründe für die Entscheidung der Kaufoptionen werden der Kostenunterschied, Zeitaufwand und persönlicher Kontakt angegeben.

Folgende Hypothesen werden aufgestellt:

\begin{enumerate}[label={H\arabic*: }]
	\item Es gibt einen positiven Zusammenhang zwischen der Häufigkeit der Bahnfahrten einer Person und dem bevorzugten Gebrauch der Selbstbedienungs-Automaten.
	\item Es gibt einen negativen Zusammenhang zwischen dem Alter einer Person und dem bevorzugten Gebrauch der Selbstbedienungs-Automaten.
	\item Der Preisunterschied wird als wichtiger empfunden als persönlicher Kontakt.
	\item Der Ticketautomat wird als schnellere Alternative zum Schalter mit persönlichem Kontakt angesehen.
\end{enumerate}


\section{Hintergrundinformation}
Die ÖBB betreibt aktuell 1032 Bahnhöfe und Haltestellen~\cite{oebbinfra_zahlendatenfakten2022} und gibt für das Jahr 2022 die Anzahl der Personenfahren mit dem Zug mit 252,5 Millionen an~\cite{oebb_zahlendatenfakten202223}.

Im Jahr 2022 wurden 50,6 Millionen Zugtickets gekauft. Die Aufteilung auf an Ticketautomaten, im bedienten Verkauf und über das Internet erworbene Tickets findet sich in Tabelle~\ref{tab:zugtickets_aufteilung}. Zu bemerken ist, dass der Prozentsatz an über das Internet und die App erworbenen Tickets im Zeitraum zugenommen hat, während der Prozentsatz der an Ticketautomaten gekauften Tickets rückläufig ist. Insgesamt ist die Anzahl der verkauften Zugtickets vom Jahr 2021 auf das Jahr 2022 angestiegen. Dies könnte mit der Lockerung von Covid-19 Maßnahmen in Österreich zusammenhängen.

\begin{table}
\begin{minipage}{\textwidth}
	\label{tab:zugtickets_aufteilung}
	\centering
	\begin{tabular}{|l|l|l|}
		\hline
		\textbf{Zugtickets} & \textbf{2021} & \textbf{2022} \\
		\hline
		\textbf{Verkaufte Tickets (pro Jahr)} & 35,4 Millionen & 50,6 Millionen \\
		\hline
		- davon Ticketautomat & 50,7\% & 41,8\% \\
		\hline
		- davon bedienter Verkauf\footnote{Ticketschalter, externe Partner, Kund:innenservice, Lounge, Geschäftskundenberater:innen, Infra-Partner, ÖBB Reisebüros} & 12,0\% & 15,5\% \\
		\hline
		- davon Internet, Smartphones, Tablets & 37,3\% & 42,7\% \\
		\hline
	\end{tabular}
\end{minipage}
	\caption{Verteilung der gewählten Kaufoptionen für Zugtickets in den Jahren 2021 und 2022~\cite{oebb_zahlendatenfakten202223}.}
\end{table}

% 613 von 1109 Bahnhöfe und Haltestellen stehen Automaten; 145 Bahnhöfe bieten Schalter an (https://konsument.at/auto-transport/bahnfahren-fahrkarten-bitte (2012))

\section{Stichprobe}
Die Grundgesamtheit dieses Interviews sind alle Personen, die in Österreich mit der ÖBB reisen und dafür Tickets bei einem Selbstbedienungs-Ticketschalter oder mit persönlicher Beratung bei einem Schalter kaufen.

Aus diesem Grund wird die Stichprobe aus Personen, die gerade den Ticketautomaten bedienen wollen oder diesen gerade bedient haben, gewählt. Es liegt eine willkürliche Stichprobenauswahl vor, daher werden in der Sektion~\ref{sec:auswertung} keine wissenschaftlichen Aussagen getroffen werden können.

Eine Zufallsstichprobe liegt nicht vor, da alle Interviews an einem einzigen Bahnhof und nur zu einem bestimmten Zeitpunkt durchgeführt werden. Die Stichprobenauswahl orientiert sich an der Relevanz, daher wird als Durchführungsort der Hauptbahnhof in Wien gewählt. Dadurch fallen allerdings alle Personen, die nicht von diesem Bahnhof abreisen aus der Stichprobe. Daher müssten für eine wissenschaftlich brauchbare Studie verschiedene Bahnhöfe in ganz Österreich als Studienort dienen. Weiters werden Personen, die seltener mit der Bahn fahren, in der Stichprobe mit hoher Wahrscheinlichkeit unterrepräsentiert.

Eine weitere Einschränkung der Stichprobe liegt durch die Interviewbereitschaft vor. Es ist möglich, dass Personen unter Zeitdruck stehen, da ihr Zug bald abfährt und sie daher kein Interview geben wollen. Die genaue Auswahl der Teilnehmer des Interviews werden von theoretischen Überlegungen des Interviewers abhängen.

\section{Gewähltes Interviewverfahren}
Das Interview wird als standardisiertes Interviewverfahren durchgeführt.

Ein quantitatives Interviewverfahren wurde hauptsächlich wegen der schwierigen Durchführbarkeit für eine einzelne Person als Interviewer ausgeschlossen. Allerdings werden einige kurze offene Fragen gestellt werden.

Als standardisiertes Interviewverfahren sind die Fragen, die meisten Antwortmöglichkeiten, sowie die Abfolge vorgegeben. Diese sind in Sektion~\ref{sec:interviewfragen} präsentiert. Die Fragen werden in einer Face-to-face Befragung erhoben.

Bei der Gestaltung der Interviewfragen, deren Antwortmöglichkeiten und des Ablaufes wurden die Fragestellungen aus Sektion~\ref{sec:fragestellungen} sowie einige best practises bedacht:
\begin{itemize}
	\item die Anzahl der Antwortmöglichkeiten liegt zwischen vier und sieben,
	\item die Fragen und Antwortmöglichkeiten sind möglichst explizit formuliert,
	\item der Ausstrahlungseffekt (Antworten auf Fragen beeinflussen sich gegenseitig) wurde berücksichtigt,
	\item leichte Fragen werden am Anfang erhoben,
	\item die Demographie wird am Ende erhoben.
\end{itemize}

Ein Situationseffekt kann in dieser Befragung nicht vermieden werden, da beispielsweise die Uhrzeit der Erhebung einen Einfluss auf die anwesenden Befragten und deren Nutzungsgewohnheiten der Bahn haben kann.


\newpage
\newgeometry{left=1.5cm,right=1.5cm, top=3cm, bottom=3cm}
\begin{landscape}

\section{Interviewfragen}
\label{sec:interviewfragen}

\begin{table}[h!]
	\large
	\begin{tabular}{|llllll|}
		\hline
		& & & & & \\
		\multicolumn{6}{|l|}{\textbf{Wie oft sind Sie im letzten Jahr in etwa durchschnittlich mit dem Zug gefahren?}}                      \\ \hline
		& $\square$ mehrmals wöchentlich & $\square$ einmal wöchentlich & $\square$ monatlich & $\square$ seltener &                        \\ \hline
		& & & & & \\
		\multicolumn{6}{|l|}{\textbf{Wo haben Sie im letzten Monat am öftesten Ihre Karte gekauft?}}                                        \\ \hline
		& $\square$ Online & $\square$ App & $\square$ Ticketautomat & \multicolumn{2}{l|}{$\square$ Schalter (persönliche Bedienung)}      \\ \hline
		& & & & & \\
		\multicolumn{6}{|l|}{\textbf{Warum haben Sie diese Variante gewählt?}}                                                              \\ \hline
		\multicolumn{6}{|l|}{}                                                                                                              \\
		& & & & & \\ \hline
		& & & & & \\
		\multicolumn{6}{|l|}{\textbf{Ist die Karte am Automat billiger als am Schalter?}}                                                   \\ \hline
		& $\square$ Ja & $\square$ Nein & $\square$ Weiß nicht &  &                                                                         \\ \hline
		& & & & & \\
		\multicolumn{6}{|l|}{\textbf{Wie viel Zeit planen Sie ein um eine Fahrkarte zu kaufen?}}                                            \\ \hline
		\multicolumn{6}{|l|}{}                                                                                                              \\
		& & & & & \\ \hline
		& & & & & \\
		\multicolumn{6}{|l|}{\textbf{Gibt es einen Unterschied beim Zeitaufwand zwischen Automat und Schalter mit persönlicher Bedienung?}} \\ \hline
		& $\square$ Ja, Automat ist schneller & $\square$ Gleich schnell & $\square$ Ja, Schalter schneller & $\square$ Nein  &             \\ \hline
		& & & & & \\
		\multicolumn{6}{|l|}{\textbf{Spielt der persönliche Kontakt mit einem ÖBB Mitarbeiter eine Rolle bei Ihrer Entscheidung?}}          \\ \hline
		& $\square$ Ja, positiv & $\square$ Ja, negativ & $\square$ Nein &  &                                                               \\ \hline
		& & & & & \\
		\multicolumn{6}{|l|}{\textbf{Bitte reihen Sie folgende Punkte bei der zukünftigen Wahl der Kaufoption:}}                            \\ \hline
		& \_\_ persönlicher Kontakt       & \_\_ Preisunterschied          & \_\_ Zeitaufwand             &  &                              \\ \hline
		& & & & & \\
		\multicolumn{6}{|l|}{\textbf{In welcher Altersgruppe liegen Sie?}}                                                                  \\ \hline
		& $\square$ 0-20 & $\square$ 21-40 & $\square$ 41-60 & $\square$ 61-80 & $\square$ 81-100                                           \\ \hline
	\end{tabular}
\end{table}

\end{landscape}
\newpage
\restoregeometry

\section{Auswertung}
\label{sec:auswertung}

TODO

\bibliography{document.bib}
\bibliographystyle{ieeetr}

\end{document}
