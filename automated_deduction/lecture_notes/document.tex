\documentclass[]{article}
\usepackage[a4paper, margin=2.5cm]{geometry}
\usepackage{amsmath}
\usepackage{amsfonts}
\usepackage{amssymb}
\usepackage{mathtools}
\usepackage{amsthm}
\usepackage[many]{tcolorbox}
\usepackage{listings}

\newtheorem{lemma}{Lemma}
\newtheorem{definition}{Definition}
\newtheorem*{definition*}{Def}
\newtheorem{algorithm}{Algorithm}

\tcolorboxenvironment{lemma}{
	colback=yellow!5!white,
	boxrule=0pt,
	boxsep=1pt,
	left=2pt,right=2pt,top=2pt,bottom=2pt,
	oversize=2pt,
	sharp corners,
	before skip=\topsep,
	after skip=\topsep,
}

\tcolorboxenvironment{definition}{
	colback=green!5!white,
	boxrule=0pt,
	boxsep=1pt,
	left=2pt,right=2pt,top=2pt,bottom=2pt,
	oversize=2pt,
	sharp corners,
	before skip=\topsep,
	after skip=\topsep,
}

\tcolorboxenvironment{definition*}{
	colback=green!5!white,
	boxrule=0pt,
	boxsep=1pt,
	left=2pt,right=2pt,top=2pt,bottom=2pt,
	oversize=2pt,
	sharp corners,
	before skip=\topsep,
	after skip=\topsep,
}
\begin{document}
	
\title{Automated Deduction Compendium SS2023}
\author{Ida Hönigmann}

\maketitle

\section{Introduction, SAT Solving}

\subsubsection{Proposition, Formulas}
\begin{definition*}[Proposition]
	Proposition is a statement that can be either true or false.
\end{definition*}

\begin{definition*}[Propositional formula, Atom, Connective]
	Atoms are boolean variables (e.g. p,q).
	
	\begin{enumerate}
		\item Atoms are formulas.
		\item $\top$, $\bot$ are formulas.
		\item If $A$ is a formula, then $\lnot A$ is a formula.
		\item If $A_1, ..., A_n$ are formulas, then $(A_1 \land ... \land A_n)$ and $(A_1 \lor ... \lor A_n)$ are formulas.
		\item If $A$ and $B$ are formulas, then $A \rightarrow B$ and $A \leftrightarrow B$ are formulas.
	\end{enumerate}

	The symbols $\top, \bot, \land, \lor, \lnot, \rightarrow, \leftrightarrow$ are called logical connectives.
\end{definition*}

\subsubsection{Precedence}
\begin{center}
	\begin{tabular}{|c c c|}
		\hline
		Connective & Name & Precedence \\
		\hline
		$\top$            & verum       &   \\  
		$\bot$            & falsum      &   \\
		$\lnot$           & negation    & 5 \\  
		$\land$           & conjunction & 4 \\
		$\lor$            & disjunction & 3 \\  
		$\rightarrow$     & implication & 2 \\
		$\leftrightarrow$ & equivalence & 1 \\  
		\hline
	\end{tabular}
\end{center}

\subsubsection{Boolean Values, Interpretation}
\begin{definition*}[Boolean values, Interpretation]
	There are two boolean vales: true ($1$) and false ($0$).
	
	An interpretation for a set $P$ of boolean variables is a mapping $I : P \rightarrow \{0,1\}$.
\end{definition*}

\subsubsection{Interpreting formulas}
\begin{enumerate}
	\item $I(\top) = 1$ and $I(\bot) = 0$
	\item $I(A_1 \land ... \land A_n) = 1$ iff $I(A_i) = 1$ for all $i$
	\item $I(A_1 \lor ... \lor A_n) = 1$ iff $I(A_i) = 1$ for some $i$
	\item $I(\lnot A) = 1$ iff $I(A) = 0$
	\item $I(A_1 \rightarrow A_2) = 1$ iff $I(A_1) = 0$ or $I(A_2) = 1$
	\item $I(A_1 \leftrightarrow A_2) = 1$ iff $I(A_1) = I(A_2)$
\end{enumerate}

\subsubsection{Safisfiable, Valid, Model}
\begin{definition*}[Satisfiable, Model, Valid]
	If $I(A) = 1$ then $I$ satisfies $A$ and $I$ is a model of $A$, denoted by $I \models A$.
	
	$A$ is satisfiable if some interpretation is a model of $A$.
	
	$A$ is valid if every interpretation is a model of $A$.
	
	$A$ and $B$ are equivalent, denoted by $A \equiv B$, if they have the same models.
\end{definition*}

\subsubsection{Connection valid, satisfiable}
\begin{enumerate}
	\item $A$ is valid iff $\lnot A$ is unsatisfiable.
	\item $A$ is satisfiable iff $\lnot A$ is not valid.
\end{enumerate}

\subsubsection{Equivalent replacement}
\begin{definition*}[Equivalent replacement]
	$A[B]$ is a formula $A$ with a fixed occurrence of subformula $B$. $A[B']$ is the formula $A$ where every occurrence of $B$ is replaced by $B'$.
\end{definition*}

\begin{lemma}[Equivalent Replacement]
	Let $I$ be an interpretation and $I \models A_1 \leftrightarrow A_2$. Then $I \models B[A_1] \leftrightarrow B[A_2]$.
	
	Let $A_1 \equiv A_2$. Then $B[A_1] \equiv B[A_2]$.
\end{lemma}


\subsubsection{Evaluating a formula}
\begin{algorithm}
\begin{lstlisting}
procedure evaluate(G,I)
input: formula G, interpretation I
output: the boolean value I(G)
begin
  forall atoms p occurring in G
    if I models p
      then replace all occurrences of p in G by 1;
      else replace all occurrences of p in G by 0;
  rewrite G into a normal form using the rewrite rules
  if G = 1 then return 1 else return 0
end
	\end{lstlisting}
\end{algorithm}

\section{Splitting, Polarities}

\section{CNF, DPLL, MiniSat}

\section{Random SAT, Horn clauses}

\section{First-Order Logic, Theories}

\section{SMT, Theory of Equality, DPLL(T)}

\section{Theory of Arrays, Theory Combination, Nelson-Oppen, Z3}

\section{First-Order Theorem Proving, TPTP, Inference Systems}

\section{Selection functions, Saturation, Fairness and Redundancy}

\section{Redundancy, First-Order Reasoning with Equality}

\section{Ground Superposition, Term Orderings}

\section{Unification and Lifting}

\section{Non-Ground Superposition}

\end{document}
